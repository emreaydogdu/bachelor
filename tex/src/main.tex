% !TeX encoding = UTF-8

%===============================================================================
% Font options are:
%   plain (default), serif (uses Palladio), sans-serif (uses Paratype Sans)
%
% Layout options are:
%   article (default, no chapters), book (for longer texts, offers \chapter)
%   Changing this value between LaTeX runs may require deleting the .aux files
%
% Paragraph options are:
%   noparskip (default, no spacing between paragraphs), parskip (spaced)
%
% Language options are:
%   de (default), en
%   Changing this value between LaTeX runs may require deleting the .aux files
%
\documentclass[serif,article,noparskip,de]{agse-thesis}

% Global parameters, replace with actual values.
\newcommand{\thesisTitle}{Sortieralgorithmen}
% -> You may use \par (but not \\) to format the title. If you do so, you'll
%    need to manually set the 'pdftitle' attribute below.
\newcommand{\studentName}{Emre Aydogdu}
%===============================================================================

\hypersetup{pdftitle={\thesisTitle}}
\hypersetup{pdfauthor={\studentName}}

\addbibresource{bibliography.bib}

% Blind texts, for demonstration only, not part of the actual template
\usepackage{lipsum}

\begin{document}

    \coverpage[
        student/id=5180123,
        student/mail=emrea97@zedat.fu-berlin.de,
        thesis/type=Bachelorarbeit,            % optional, default: Bachelorarbeit
        thesis/group={Arbeitsgruppe Technische Informatik}, % optional, default: AGSE
        thesis/advisor={Matt Visor},           % optional
        thesis/examiner={Prof. Dr. Mia Maus},
        thesis/examiner/2={Prof. Dr. Bob Bär}, % optional
        thesis/date=\today,                    % optional, default: \today
    %title/size=\LARGE,      % set this value to overwrite automatic font size
    %abstract/separate       % toggle this to move the abstract to its own page
    ]
    { % Your abstract here:
        \noindent
        \lipsum[1]
    }

    \include{declaration}

    \cleardoublepage

    \tableofcontents

    \cleardoublepage

    \pagestyle{fancy}

% Actual content starts here

    \input{1_introduction}
    \input{2_fundamentals}
    \input{3_main}
    \input{4_conclusion}

    \printbibliography

    \appendix
    \include{5_appendix}

\end{document}
